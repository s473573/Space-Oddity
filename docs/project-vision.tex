\documentclass[a4paper,12pt]{article}

\usepackage{titlesec}
\usepackage{enumitem}
\usepackage{geometry}

\geometry{a4paper, margin=1in}

\title{\Space Oddity\\ Project Vision Document}
\author{Eduard Pokhylko & Co.}
\date{\today}

\begin{document}
\maketitle

\section*{Introduction}
Briefly introduce the Interstellar Odyssey project, its objectives, and the team members involved.

\section*{Project Overview}
Provide an overview of the interactive 3D space flight simulator. Describe the main features, goals, and target audience.

\section*{Application Theme}
Outline the theme of the application, including the storyline, scenes, and ways of interacting with the environment.

\section*{Graphics Techniques}
Detail how advanced computer graphics techniques, such as quaternions, normal mapping, PBR, bloom, parallel transport frames, and others, will be incorporated into the project.

\section*{Chosen CG Method}
Specify the additional CG method chosen for achieving 100% marks and its relevance to the project.

\section*{Presentation Topic}
Briefly describe the CG method or your team's proposed idea for the 7-minute presentation to the class.

\section*{Visual Appeal}
Highlight the focus on achieving visual appeal through realistic lighting, shadows, reflections, and high-quality textures.

\section*{Interactivity}
Describe how the simulator will offer engaging gameplay, including dynamic ship control, interaction with the environment, and potential missions or challenges.

\section*{Application Correctness and Complexity}
Outline the criteria for evaluating the correctness and complexity of the application code, emphasizing the correct implementation of computer graphics methods.

\section*{Timeline}
Provide a high-level timeline for the project, indicating major milestones and deadlines.

\section*{Conclusion}
Summarize the overall vision for the Interstellar Odyssey project, emphasizing its uniqueness and potential impact.

\end{document}

